\documentclass[12pt]{article}
% \usepackage[T1]{fontenc}
\renewcommand{\familydefault}{\sfdefault}
\usepackage{helvet}
% \usepackage{cmss} 
\usepackage{graphics,graphicx,amssymb,amsmath,enumerate}
\usepackage{fancyhdr}
\usepackage[nodayofweek]{datetime}
\usepackage{lastpage}
\usepackage{etaremune}

\usepackage{xcolor} % Add this package
\usepackage{hyperref}

% Define a custom color
\definecolor{darkgray}{RGB}{51, 51, 51} % Equivalent to #333333
% Define a custom color for #555555
% \definecolor{darkgray}{RGB}{85, 85, 85}
% Define the colors
\definecolor{navyblue}{HTML}{003366}
\definecolor{brightblue}{HTML}{0033CC}

\hypersetup{
    colorlinks=true,
    linkcolor=navyblue, % Use navy blue
    filecolor=magenta,
    urlcolor=brightblue, % Use bright blue
}

% \hypersetup{
%     colorlinks=true,
%     linkcolor=blue,
%     filecolor=magenta,
%     urlcolor=darkgray, % Use the custom defined color
% }
 
\urlstyle{same}

\usepackage[super]{nth}

\setlength{\textwidth}{6.5in} 
\setlength{\textheight}{9.4in}
\setlength{\topmargin}{-0.7in} 
\setlength{\oddsidemargin}{0in}
\setlength{\evensidemargin}{0in} 

% \setlength{\topmargin}{0in} % Set top margin to 0 inches
% \setlength{\headheight}{0in} % Remove any header height (optional)
% \setlength{\headsep}{0.5in} % Set space between header and text (optional)


\newdateformat{monthyeardate}{\monthname[\THEMONTH] \THEYEAR}

\begin{document}
%\pagenumbering{gobble}% Remove page numbers (and reset to 1)
\pagestyle{fancy}
\fancyhf{}
\renewcommand{\headrulewidth}{0pt}

\rhead{\footnotesize{\thepage\ of \pageref{LastPage}}}
\lfoot{\footnotesize{L.D. Vega}}
\rfoot{\footnotesize{Updated \monthyeardate\today}} 

\begin{center} 
\bfseries{
\LARGE \uppercase{Laura D. Vega} \\
\large \uppercase{Curriculum Vitae}}
\end{center}

\noindent
% \begin{center}{\bf Astrophysicist \(|\) Heising-Simons Astrophysics Postdoctoral Launch Program Fellow} \\
\begin{center}{\bf Heising-Simons Astrophysics Postdoctoral Launch Program Fellow} \\
% Vanderbilt University - Department of Physics \& Astronomy}
\end{center}

{ %\small
% \noindent Department of Astronomy \hfill \textit{Email:} ldvega@umd.edu \\
% \noindent University of Maryland, College Park \hfill \href{https://www.astro.umd.edu/people/directory.html#Postdocs\%20and\%20Faculty\%20Assistants}{UMD Astro Directory}  \\
% 4296 Stadium Drive, College Park, MD 20742 \hfill Personal: \href{https://lunaalphalyrae.github.io/}{https://lunaalphalyrae.github.io/}
% % 6301 Stevenson Center, Nashville, TN 37235 \hfill Citizenship: United States of America \\
% %  NASA Goddard Space Flight Center \hspace{7.0cm} \url{http://my.vanderbilt.edu/lauradvega}
% \vskip 0.1in
% \noindent Exoplanets and Stellar Astrophysics Laboratory \hfill \textit{Email:} laura.d.vega@nasa.gov
% \\ 
% NASA Goddard Space Flight Center \hfill
% \url{https://science.gsfc.nasa.gov/sed/bio/laura.d.vega} \\
% Mail Code 667, 8800 Greenbelt Rd, Greenbelt, MD 20771
% \vskip 0.1in
% \noindent Center for Research and Exploration in Space Science and Technology II (CRESST II) \\ NASA/GSFC, Greenbelt, MD 20771 \\


\begin{center}
\href{https://orcid.org/0000-0002-5928-2685}{ORCID}: 0000-0002-5928-2685 \textbar\, \href{https://ui.adsabs.harvard.edu/public-libraries/VvvdxodpROKdCd9E7G0Mkg}{ADS} \textbar\, \href{https://arxiv.org/search/?query=Vega%2C+L.+D.&searchtype=author&abstracts=show&order=-announced_date_first&size=50}{Astro-PH}
\end{center}

% \noindent Research Mentorship Network: Jackie Villadsen (Bucknell; current); Elisa Quintana (Goddard; current); \\ Allison Youngblood (Goddard; current); and Padi Boyd (NASA HQ; 2021-2024)\\

\noindent University of Maryland, College Park \hfill \textit{Email:} ldvega@umd.edu \\
\noindent Department of Astronomy \hfill \\
% \href{https://www.astro.umd.edu/people/directory.html#Postdocs\%20and\%20Faculty\%20Assistants}{UMD Astro Directory} \hfill \\
4296 Stadium Drive, College Park, MD 20742 \hfill \\
Personal website: \href{https://LunaAlphaLyrae.github.io/}{https://LunaAlphaLyrae.github.io/}
\vskip 0.1in
\noindent NASA Goddard Space Flight Center (GSFC) \hfill \textit{Email:} laura.d.vega@nasa.gov \\ 
Exoplanets and Stellar Astrophysics Laboratory \\ \& Center for Research and Exploration in Space Science and Technology II (CRESST II) \hfill \\
Mail Code 667, 8800 Greenbelt Rd, Greenbelt, MD 20771 \\
Professional Website: \href{https://science.gsfc.nasa.gov/sed/bio/laura.d.vega}{https://science.gsfc.nasa.gov/sed/bio/laura.d.vega} \\
% \vskip 0.08in
% \noindent Center for Research and Exploration in Space Science and Technology II (CRESST II) \\ NASA/GSFC, Greenbelt, MD 20771 \\


%6301 Stevenson Center \hspace{4.35cm} 
% Nashville, TN 37235 \hspace{4.72cm}
% \bigskip
}
\noindent
{\bf Summary} \\
\vspace{-10mm}
\begin{center}
\rule{\textwidth}{0.2mm}
\end{center}
\vspace{-3mm}

% Astrophysics Ph.D. candidate at Vanderbilt University stationed at NASA-GSFC. 
% My research interests include multiwavelength astronomy, late-stage stellar evolution, post-AGB binaries, circumbinary disks, and X-ray/UV flares from M-dwarf stars. \\

% \noindent Resourceful astrophysicist with a PhD and a proven track record in leading and contributing to multiwavelength data analysis and research, resulting in published scientific papers. My research interests include multiwavelength astronomy, late-stage stellar evolution, post-AGB binaries, circumbinary disks, and X-ray/UV flares from M-dwarf stars. Recognizing the importance of effective communication in science, I have engaged with Toastmasters International since my undergraduate years to enhance my public speaking, leadership, and mentoring skills. My lifelong enthusiasm for astronomy drives my passion for sharing knowledge and building connections within the community while prioritizing communication, collaboration, adaptability, and productive working relationships within teams. \\
\noindent Astrophysicist with a PhD specializing in multiwavelength astronomy, with a focus on M-dwarf stellar flares (postdoctoral work) and late-stage stellar evolution (dissertation). Experienced in leading and contributing to scientific publications, demonstrating adaptability in research focus. Skilled in data analysis, public speaking, and outreach. Passionate about sharing knowledge and collaborating on diverse research projects to advance our understanding of stellar physics. \\

% \bigskip

\noindent
{\bf Employment \& Education} \\
\vspace{-10mm}
\begin{center}
\rule{\textwidth}{0.2mm}
\end{center}
\vspace{-3mm}
% \noindent
\textbf{\textit{Heising-Simons Post-doctoral Fellow}}  \hfill  May 2021 -- Apr. 2025 \\
\hspace*{1cm}The University of Maryland, College Park \& NASA Goddard Space Flight Center\\
\textbf{\textit{NASA Harriett G. Jenkins Pre-doctoral Fellow}} \hfill  Aug. 2015 -- Apr. 2020\\
\hspace*{1cm}Fisk University \& Vanderbilt University \\
\noindent\textit{\textbf{Fisk-Vanderbilt Masters-to-PhD Bridge Program Fellow}} \hfill  Aug. 2014 -- Apr. 2021 \\
\hspace*{1cm}Fisk University \& Vanderbilt University \\
\textbf{\textit{Undergraduate Research Assistant}} \hfill  May 2010 -- Jul. 2014 \\
\hspace*{1cm}The University of Texas at San Antonio (UTSA) \\
% \vspace{-7mm}
% \begin{itemize}

% \noindent
% {\bf Education} \\
% \vspace{-10mm}
% \begin{center}
% \rule{\textwidth}{0.2mm}
% \end{center}
% \vspace{-3mm}
% % \noindent
\noindent\textbf{Astrophysics Ph.D., 2021} -- Vanderbilt University \hfill \textit{Advisor:} Keivan G. Stassun \\
\hspace*{1cm}Dissertation Title: {\it Stellar Evolution at the Crossroads: Resolving the Nature of RV Tauri Variable Systems Using Unprecedented Observations from the Kepler and XMM-Newton Space Telescopes} \\
\textbf{Physics M.A., 2017} -- Fisk University \hfill \textit{Advisor:} Keivan G. Stassun \\
\hspace*{1cm}Thesis Title: {\it Evidence for Binarity and Disk Obscuration in Kepler Observations of the Pulsating RV Tau Variable Star DF Cygni} \\
% \end{itemize}
% \textbf{Astrophysics Ph.D., 2021} -- Vanderbilt University \hfill \textit{Advisor:} Keivan G. Stassun \\
% Dissertation Title: {\it Stellar Evolution at the Crossroads: Resolving the Nature of RV Tauri Variable Systems Using Unprecedented Observations from the Kepler and XMM-Newton Space Telescopes} \\[2mm]
% \textbf{Physics M.A., 2017} -- Fisk University \hfill \textit{Advisor:} Keivan G. Stassun \\
% Thesis Title: {\it Evidence for Binarity and Disk Obscuration in Kepler Observations of the Pulsating RV Tau Variable Star DF Cygni} \\[2mm]
% \vspace{-7mm}
% \begin{itemize}
\textbf{Physics B.Sc., 2013} -- The University of Texas at San Antonio \hfill \textit{Advisor:} Eric M. Schlegel \\
% \end{itemize}

% \noindent
% {\bf Relevant Skills} \\
% \vspace{-10mm}
% \begin{center}
% \rule{\textwidth}{0.2mm}
% \end{center}
% \vspace{-3mm}
% \noindent
% \textbf{Languages}: English; Spanish\\
% \textbf{Computer (varying levels of proficiency)}: Unix; Python; LaTeX; Lightkurve; HEASARC: FTOOLS, XSPEC; SAOImage DS9 \\
% %X-ray Spectral Fitting Package (XSPEC), Chandra Interactive Analysis of Observations (CIAO) \\
% \textbf{Data experience (varying levels of proficiency)}: X-ray/Ultraviolet (NICER, {\it Swift}); Optical ({\it Kepler}, TESS, AAVSO, DASCH) \\
% % \indent Space Telescopes: \\ X-ray/UV: Chandra, Swift, Kepler, and the Transiting Exoplanet Survey Satellite. \\ \indent Ground-based: \\ American Astronomical Variable Stars Observers, Digital Access to a Sky Century @ Harvard. \\ \indent Interferometry: \\ The Smithsonian Sub-millimeter Array (Radio) \\
\newpage
\noindent
{\bf Relevant Skills} \\
\vspace{-10mm}
\begin{center}
\rule{\textwidth}{0.2mm}
\end{center}
\vspace{-3mm}
\noindent
\textbf{Languages}: English; Spanish\\
\textbf{Computer}: Unix; Python; LaTeX; Astropy; Lightkurve; HEASARC:FTOOLS, XSPEC; \\ SAOImage DS9 \\
\textbf{Data experience}: X-ray/Ultraviolet (NICER, {\it Swift}); Optical ({\it Kepler}, TESS, AAVSO, DASCH), and Radio (VLA) time series \\
\textbf{Writing Skills}: Technical writing; Proposal writing \\

% \newpage
\noindent
{\bf Grants/Fellowships} \\
\vspace{-10mm}
\begin{center}
\rule{\textwidth}{0.2mm}
\end{center}
\vspace{-3mm}
\noindent
Heising-Simons Foundation Future Faculty Launch Program Postdoctoral Fellowship \hfill 2021 - 2025 \\
NASA Sellers Exoplanet Environments Collaboration (SEEC) Internal Scientist Funding Model (ISFM) Mini-Proposals (Role: \textbf{Co-I}; PI: A. Mandell, Total Budget: \$50K)  \hfill 2024 - 2025 \\ 
% NASA Swift Cycle 18 GI Program (PI: W. Howard - \$35K) \hfill 2022 - 2023 \\
NASA Astrophysics Data Analysis (ADAP) (Role: \textbf{Co-I}; PI: J. Schlieder, Total Budget:  \$62K) \hfill 2021 - 2023 \\
NASA TESS Cycle 4 GI Program (Role: {\bf PI}, Total Budget: \$70K) \hfill 2021 - 2022 \\
NASA TESS Cycle 3 GI Program (Role: {\bf PI}, Total Budget: \$50K) \hfill 2020 - 2021 \\
% NASA Swift Cycle 16 GI Program (PI: A. Youngblood - \$35K) \hfill 2020 - 2021 \\
NASA OSTEM Minority University Research Education Project (MUREP)/Advanced STEM Training and Research (ASTAR): Harriett G. Jenkins Pre-doctoral Fellowship \hfill 2015 - 2020 \\
Smithsonian Latino Center - LIP (Research at the Center for Astrophysics and Outreach at the National Air \& Space Museum) \hfill 2017, 2018, 2019 \\
UTSA Vaughan Family Grant (AAS Conference Travel) \hfill 2011, 2012, 2014 \\
National Institute on Minority Health and Health Disparities – Nanotechnology and Human Health Core Grant (2012 Summer Undergraduate Nanoparticles Research at UTSA) \hfill 2012 \\
Bill \& Alicia Hoover Grant (For 2010 Summer Research at UTSA) \hfill 2010 \\

% \bigskip
% \newpage
\noindent
{\bf Service} \\
\vspace{-10mm}
\begin{center}
\rule{\textwidth}{0.2mm}
\end{center}
\vspace{-3mm}
\noindent
NASA Fellows and Scholars Alumni at the Southern Regional Education Board (SREB) Institute on Teaching and Mentoring, Panelist \hfill 2018, 2019, 2023 \\
NASA Astrophysics Theory Program (ATP), Review Panel Member \hfill Sep. 2021 \\
% - Evaluated proposals and contributed to the selection process. \\
NASA Fellowship Activity Peer Review Panel, Executive Scribe \hfill Apr. 2020 \\
NASA TESS18 Review Panel, Scribe \hfill May 2019 \\
NASA en Español, English-to-Spanish Text Translator (occasional) \hfill 2018 - Present \\

% - Translated and adapted scientific materials and outreach content into Spanish, ensuring accessibility and clarity for Spanish-speaking audiences. 
% Collaborated with a diverse team to enhance NASA’s engagement with the Hispanic community, fostering a greater understanding of NASA's missions and research initiatives. \\




\noindent
{\bf Honors and Awards} \\
\vspace{-10mm}
\begin{center}
\rule{\textwidth}{0.2mm}
\end{center}
\vspace{-3mm}
\noindent
Most Outstanding Student Publication - Vanderbilt Physics \& Astronomy Department \hfill 2018 \\
NASA Office of Education's MUREP: Harriett G. Jenkins Pre-doctoral Fellow \hfill 2015 \\
Best Poster in Physics - Fisk University Science Symposium \hfill 2016 \\
Toastmasters International - Competent Leader \hfill 2015 \\
Toastmasters International - Competent Communicator \hfill 2015 \\
The Sam Madrid Jr. Endowed Scholarship - UTSA \hfill 2012 \\
The Dr. B. Thyagarajan Endowed Scholarship - UTSA \hfill 2009 \\
Dean’s List - UTSA \hfill 2007, 2008 \\
Honor Roll - UTSA \hfill 2009, 2010, 2011 \\

\noindent
{\bf Publications} ([Number] = Number of citations for refereed papers as of CV's date-of-update) \\
\vspace{-10mm}
\begin{center}
\rule{\textwidth}{0.2mm}
\end{center}
\vspace{-3mm}
\noindent
% \begin{enumerate}[\bfseries 1.] 
\begin{etaremune}
\renewcommand\labelenumi{\bfseries\theenumi .}
\item {\bf Vega L. D. et al., In Prep.}; {\it Simultaneous TESS and Swift Observations of Wolf~359}.

\item {\bf Vega L. D. et al. 2024, In Prep.}; {\it From Radio to X-ray: Tracing the flares of YZ Canis Minoris Across the Spectrum}.

\item {Howard, W. S. et al. 2024;}
\href{https://ui.adsabs.harvard.edu/abs/2025AJ....169...27H/abstract}{\it Preparing for the Early eVolution Explorer: Characterizing the Photochemical Inputs and Transit Detection Efficiencies of Young Planets Using Multiwavelength Flare Observations by TESS and Swift}, The Astronomical Journal, 169, 27.

\item {Paudel R. R. et al. 2024}; 
\href{https://ui.adsabs.harvard.edu/abs/2024ApJ...971...24P/abstract}{\it A Multiwavelength Survey of Nearby M Dwarfs: Optical and Near-ultraviolet Flares and Activity with Contemporaneous TESS, Kepler/K2, {\it Swift}, and HST Observations}, The Astrophysical Journal, 971, 24. [7]

\item {Tovar-Mendoza, G. et al. 2023};
\href{https://ui.adsabs.harvard.edu/abs/2023arXiv230705806T/abstract}{\it Enabling Stellar Flare Science in the Roman Galactic Bulge Survey: Cadence, Filters, and the Read-Out Strategy Matter}, Roman Core Community Survey White Paper. [1]

\item {Wittrock, J. M. et al. 2023};
\href{https://ui.adsabs.harvard.edu/abs/2023arXiv230204922W/abstract}{\it Validating AU Microscopii d with Transit Timing Variations}, The Astronomical Journal, 166, 232. [16]

\item {Barclay, T. et al. 2023, Submitted to AAS Journals}; \href{https://ui.adsabs.harvard.edu/abs/2023arXiv230110866B/abstract}{\it The transmission spectrum of the potentially rocky planet L 98-59 c}. [12]

\item {Chavali, S. et al. 2022}; \href{https://ui.adsabs.harvard.edu/abs/2022RNAAS...6..201C/abstract}{\it A Pilot Survey of an M Dwarf Flare Star with Swift’s UV Grism}, Research Notes of the AAS, 6, 201. [6]

\item {Wittrock, J. M. et al. 2022}; \href{https://ui.adsabs.harvard.edu/abs/2022AJ....164...27W/abstract}{\it Transit Timing Variations for AU Microscopii b and c}, The Astronomical Journal, 164, 27. [16]

\item Gilbert, E. A. et al. 2022; \href{https://ui.adsabs.harvard.edu/abs/2022AJ....163..147G/abstract}{\it Flares, Rotation, and Planets of the AU Mic System from TESS Observations}, The Astronomical Journal, 163, 147. [55]

\item {Paudel, R. R. et al. 2021}; 
\href{https://ui.adsabs.harvard.edu/abs/2021ApJ...922...31P/abstract}{\it Simultaneous Multiwavelength Flare Observations of EV Lacertae}, The Astrophysical Journal, 922, 31. [30]

\item {\bf Vega, L. D. et al. 2021}; \href{https://ui.adsabs.harvard.edu/abs/2021ApJ...909..138V/abstract}{\it Multiwavelength Observations of the RV Tauri Variable System U Monocerotis: Long-Term Variability Phenomena Can Be Explained by Binary Interactions With a Circumbinary Disk}, The Astrophysical Journal, 909, 138. (\href{https://www.nasa.gov/feature/goddard/2021/scientists-sketch-aged-star-system-using-over-a-century-of-observations/}{\bf NASA Feature}). [5]

\item Gilbert, E. A. et al. 2020; \href{https://ui.adsabs.harvard.edu/abs/2020AJ....160..116G/abstract}{\it The First Habitable Zone Earth-sized Planet from TESS. I: Validation of the TOI-700 System}, The Astronomical Journal, 160, 116.
(\href{https://www.nasa.gov/feature/goddard/2020/nasa-planet-hunter-finds-its-1st-earth-size-habitable-zone-world}{\bf NASA Feature}). [104]

\item Kostov, V. B. et al. 2019; \href{https://ui.adsabs.harvard.edu/abs/2019AJ....158...32K/abstract}{{\it The L 98-59 System: Three Transiting, Terrestrial-size Planets Orbiting a Nearby M Dwarf}}, The Astronomical Journal, 158, 32.
(\href{https://www.nasa.gov/feature/goddard/2019/nasa-s-tess-mission-finds-its-smallest-planet-yet}{\bf NASA Feature}). [111]

\item {\bf Vega, L. D. et al. 2017}; \href{https://ui.adsabs.harvard.edu/abs/2017ApJ...839...48V/abstract}{{\it Evidence for Binarity and Possible Disk Obscuration in Kepler Observations of the Pulsating RV Tau Variable DF~Cygni}}, The Astrophysical Journal, 839, 48. [14]

\item Schlegel, E. M. et al. 2016; \href{https://ui.adsabs.harvard.edu/abs/2016ApJ...823...75S/abstract}{{\it NGC 5195 In M51: Feedback ‘Burps’ After a Massive Meal?}}, The Astrophysical Journal, 823, 75. (\href{https://www.nasa.gov/mission_pages/chandra/nasa-s-chandra-finds-supermassive-black-hole-burping-nearby.html}{\bf NASA Feature}). [11]

\end{etaremune}
% \end{enumerate}

% \bigskip
% \newpage
\noindent
% \vspace{-3mm}
{\bf Leadership \& Outreach} \\
\vspace{-10mm}
\begin{center}
\rule{\textwidth}{0.2mm}
\end{center}
\vspace{-3mm}

\noindent
\vspace{-3mm}
{\bf Public Talks \& Engagement} \\
\vspace{-13mm}
\begin{center}
% \rule{\textwidth}{0.2mm}
\end{center}
\vspace{-3mm}
\begin{itemize}
    \item  \textbf{\href{https://youtu.be/DMUk448-734?si=yv1RBLK9qMrYYNFq}{We asked a NASA expert (Spanish)} - Speaker \hfill Apr. 2024}\\
    Recorded an educational video with @NASAenEspañol titled: ``\textit{¿Qu\'e tan grande es el espacio?} (Just how large is space?)."
    \item \textbf{\href{https://astronomyontap.org/event/astronomy-on-tap-baltimore-january-edition/}{Astronomy on Tap (AoT) Baltimore}, Guilford Hall Brewery, Baltimore, MD \hfill Jan. 2024} \\
    General Public Talk: ``\textit{Stellar Cheers: A Toast to the Eccentricities of RV Tauri Binary Stars}.''
    \item \textbf{\href{https://www.spaceappschallenge.org/2023/locations/hermosillo/?tab=schedule}{NASA International Space Apps Challenge Hermosillo}, Tecnol\'{o}gico de Monterrey Campus Sonora Norte, Hermosillo, Sonora, M\'{e}xico - Local Mentor and Judge \hfill Oct. 2023} \\
    Engaged with student groups, evaluated presentations, and gave a talk on my academic journey and research. Delivered a formal scientific talk to professors and graduate students at the university.
    \item \textbf{\href{https://www.twitch.tv/videos/1316037901}{The Roman Space Observer} \hfill Jun. 2022} \\
    Twitch stream (Spanish version) interview with @NASAenEspa\~nol to promote NASA's retro 8-bit Roman Space Observer video game and to discuss some of the science The Nancy Grace Roman Space Telescope will do.
    \item \textbf{NASA Hyperwall, Honolulu, HI \hfill  Jan. 2020} \\ ``{\it Astrophysics with TESS}" presentation at American Astronomical Society, AAS 235th Meeting.
\end{itemize}

\noindent
\vspace{-3mm}
{\bf Committee Work \& Mentoring} \\
% \subsection*{Committee Work \& Mentoring}
\vspace{-13mm}
\begin{center}
% \rule{\textwidth}{0.2mm}
\end{center}
\vspace{-3mm}
\begin{itemize}
    \item \textbf{Graduate Resources Advancing Diversity with Maryland Astronomy and Physics (\href{https://www.umdgradmap.org/}{GRAD -MAP}) Winter Workshop - Presenter \hfill 2021, 2022, 2023, 2024} \\
    Presentations to GRAD-MAP student about my career path and annual informational presentation on ``{\it Post Bacs, Masters, and Bridge Programs}."
    \item \textbf{\href{https://www.astro.umd.edu/events/colloquia/bang.html}{Better Astronomy for the New Generation! (BANG!)} - Organizing Committee Member \hfill Sept. 2022--May 2023} \\
    Organized seminars on equity, career paths, and soft skills for astronomers.
    \item \textbf{Research Mentor \hfill Oct. 2020--Jan. 2021} \\
    Mentored a high school senior in Madison County, VA, on classifying M-dwarf flares based on the Solar-flare classification system.
    \item \textbf{Fisk-Vanderbilt Bridge Program at SACNAS, Long Beach, CA - Recruiter \hfill Oct. 2016} \\
    Booth at SACNAS to recruit undergraduate student for the Fisk-Vanderbilt Bridge Program.
    \item \textbf{DCA Stars at The University of Maryland College Park - Member \hfill Summer 2016} \\
    Students and mentors met weekly and discussed our research, goals, setbacks, and successes.
    \item \textbf{Inclusive Astronomy Conference at Vanderbilt, Nashville, TN \hfill Jun. 17--19, 2015} \\
    Volunteer host member for the inaugural conference on diversity and inclusion.
    \item \textbf{Astronomy Torus 2012 Workshop at UTSA, San Antonio, TX \hfill Dec. 5--7, 2012} \\
    International workshop focused on active galactic nuclei; Local Organizing Committee Member: Logistics planning; Registration desk.
\end{itemize}

\noindent
\vspace{-3mm}
{\bf Youth Outreach \& Educational Events} \\
\vspace{-10mm}
\begin{center}
% \rule{\textwidth}{0.2mm}
\end{center}
\vspace{-3mm}
\begin{itemize}
    \item \textbf{Ed Cody Elementary STREAM Night - Community Event \hfill Jan. 2025} \\
    Engaged students and parents in an interactive Astronomy Chat at this community event, using a projector to showcase cosmic images while discussing planets, galaxies, and the universe at Cody Elementary in San Antonio, TX.
    \item \textbf{Society of Physics Students and SACNAS student chapters \hfill Feb. 2023} \\
    \textbf{Invited Talk}: \textit{My Journey to the PHD}; San Antonio College, San Antonio, TX.
    \item \textbf{Colorado Springs Cool Science Festival - Zoom-a-Scientist \hfill Oct. 2020} \\
    Virtual sessions with elementary students in Fairbanks, AK and Colorado Springs, CO, discussing astronomy.
    \item \textbf{Smithsonian National Air \& Space Museum - Volunteer Astronomer \hfill Oct. 2017 - Present} \\
    Engaged in \href{https://airandspace.si.edu/events/astronomy-chat-gabriella-alvarez-and-laura-vega-0}{Astronomy Chats} and Hispanic Heritage Month Outreach Familia Day in Washington, DC.
    \item \textbf{Barrera Veterans Elementary's Career Day - Astronomy Chat \hfill May 2019} \\
    Shared experiences in astronomy with students across different grade levels in Von Ormy, TX.
    \item \textbf{Bancroft Elementary Saturday Academy - Astronomy Chat \hfill Jan. 2019} \\
    Visited bilingual Latinx students to discuss astronomy, in collaboration with the Smithsonian Air \& Space Museum and sponsored by the Smithsonian Latino Center in Washington, DC.
    \item \textbf{Smithsonian Astrophysical Observatory's Rising Stars Camp - Astronomy Chat  \hfill Jul. 2018} \\
    Led activities using robotic telescopes and gave a research talk on star lifecycles at St. Katharine Drexel in Roxbury Crossing, MA.
\end{itemize}

\newpage

\noindent
\vspace{-3mm}
{\bf Other Contributions \& Leadership} \\
\vspace{-13mm}
\begin{center}
% \rule{\textwidth}{0.2mm}
\end{center}
\vspace{-3mm}
\begin{itemize}
    \item \textbf{\href{https://umdlatino.wordpress.com/about/board-members/}{Latinx Employee Association (LEA)} at the University of Maryland - Grad Student and Postdoc Representative \hfill Feb. 2022 - Present} \\
    Collaborating in the revitalization and re-establishment of the association to support Latino/ Hispanic employees through community building, advocacy, professional development, and outreach initiatives.
    \item \textbf{Toastmasters International \hfill Oct. 2013 – Present} \\
    \textbf{Vice President of Public Relations} at Goddard Toastmasters (06/2022 – 2024) and Vanderbilt Toastmasters (05/2015 – 2018); \textbf{Treasurer} at UTSA Roadrunner Toastmasters (01/2014 – 07/2014). As a member, I deliver prepared and extemporaneous speeches, evaluate other members’ speeches, and take on \textbf{leadership roles} during meetings.
    \item \textbf{Vanderbilt SACNAS \hfill Jan. 2015 - May 2017} \\
    \textbf{Founding officer (Treasurer)} and \textbf{Public Relations Officer} for the first Society for Advancement of Chicanos \& Native Americans in Science chapter in Tennessee at Vanderbilt University. I also helped organized the Vanderbilt SACNAS Science Cafe.
    \item \textbf{The Physics Honor Society \text{\textbar} Sigma Pi Sigma at UTSA \hfill Apr. 2013 - Jan. 2014} \\
    \textbf{Founding officer (President)} for the The Physics Honor Society chapter at UTSA. I helped organize study sessions for the Physics GRE and tutored students in physics and math.
    \item \textbf{Society of Physics Students at UTSA \hfill Aug. 2012 - Apr. 2013} \\
    \textbf{Public Relations Officer}: Created promotional materials, managed social media, and organized monthly outreach events like UTSA’s “\textit{Friday Nights, Celestial Lights}." I also helped lead various STEM outreach events for high school and elementary school students.
\end{itemize}

% \newpage
\noindent
{\bf ACCEPTED Observing Proposals} \\
\vspace{-10mm}
\begin{center}
\rule{\textwidth}{0.2mm}
\end{center}
\vspace{-3mm}
\noindent
% \begin{enumerate}[\bfseries 1.] 
\begin{etaremune}
\renewcommand\labelenumi{\bfseries\theenumi .}

\item TESS Cycle 6 GI Program \hfill Jan. 2024 \\
\textit{Study Of Optical/Near-Ultraviolet Energy Fractionation In Stellar Flares Using Simultaneous 20-s TESS And Swift NUV Data} \\ Role: \textbf{Co-I}, PI: Rishi Paudel

\item ALMA Cycle 9 GI Program \hfill Apr. 2022 \\
\textit{The Origin and Impact of Flares in M Dwarf Systems} \\ Role: \textbf{Co-I}, PI: Meredith MacGregor

\item TESS Cycle 5 GI Program \hfill Jan. 2022 \\
\textit{Multiwavelength TESS-Swift-NICER Observations of Pulsations in Flares on Solar-Type Stars} \\ Role: \textbf{Co-I}, PI: Teresa Monsue

\item Swift Cycle 18 GI Program \hfill Sep. 2021 \\
\textit{A Swift and Alma View of the Origin and Impact of M-Dwarf Flares} \\ Role: \textbf{Co-I}, PI: Ward Howard; Total Budget: \$35K

\item TESS Cycle 4 GI Program \hfill Jan. 2021 \\
\textit{Exploring the Star-planet Connection via Simultaneous Tess and Swift Observations of Highly Active M Dwarfs} \\ Role: \textbf{PI}, PI: \textbf{Vega, L. D.}; Total Budget: \$70K

\item TESS Cycle 4 GI Program \hfill Jan. 2021 \\
\textit{Using Tess 20-S Cadence Data To Study Flares On M Dwarfs} \\ Role: \textbf{Co-I}, PI: Rishi Paudel; Total Budget: \$70K

\item NICER Cycle 3 GI Program \hfill Nov. 2020 \\
\textit{A Study of M Dwarf Flares Using Simultaneous High Cadence Multi-wavelength Data} \\ Role: \textbf{Co-I}, PI: Rishi Paudel

\item XMM-Newton - Announcement of Opportunity 20 \hfill Oct. 2020 \\
\textit{Pilot Study of RV Tau Variables: A new class of X-ray emitting stars?} \\ Role: \textbf{Co-I}, PI: Rodolfo Montez Jr.

\item Hubble Space Telescope Cycle 28 GO Program \hfill Oct. 2020 \\
\textit{Confirming a Tentative Detection of an Atmosphere Around a Potentially Rocky Planet} \\ Role: \textbf{Co-I}, PI: Thomas Barclay

\item Swift Cycle 16 GI Program  \hfill Sep. 2020 \\
\textit{Defining the Energy Budget for Abiogenisis on M-dwarf Planets} \\ Role: \textbf{Co-I}, PI: A. Youngblood; Total Budget: \$35K

\item Swift - Target of Opportunity \hfill Jul. 2020 \\
\textit{20 kilo-second UVOT observations of the AU~Mic system} \\ Role: \textbf{PI}, PI: \textbf{Vega, L. D.}

\item Joint TESS-Swift Cycle 3 GI Program  \hfill Jan. 2020 \\
\textit{Exploring the Star-Planet Connection via Simultaneous TESS and Swift Observations of Highly Active M Dwarfs} \\ Role: \textbf{PI}, PI: \textbf{Vega, L. D.}; Total Budget: \$50K 

\item TESS Cycle 3 GI Program  \hfill Jan. 2020 \\
\textit{Discovering Circumbinary Planets With Tess} \\ Role: \textbf{Co-I}, PI: Veselin Kostov; Total Budget: \$50K

\item NICER Cycle 2 GI Program \hfill Nov. 2019 \\
\textit{Multiwavelength Observations of Highly Active M Dwarfs} \\ Role: \textbf{Co-I}, PI: Rishi Paudel

\item Hubble Space Telescope Cycle 27 \hfill Apr. 2019 \\
\textit{Searching for Secondary Atmospheres in a System of Benchmark Worlds} \\ Role: \textbf{Co-I}, PI: Thomas Barclay

\item NOAO - Center for High Angular Resolution Astronomy \hfill Apr. 2019\\
\textit{Diving into the close stellar environment of the magnetic
RV Tauri star U~Mon} \\ Role: \textbf{Co-I}, PI: Laurence Sabin

\item SAO - Submillimeter Array \hfill Mar. 2019 \\
 \textit{Hunting for Disks Around Pulsating RV Tau Stars} \\ Role: \textbf{PI}, PI: \textbf{Vega, L. D.}

\item SAO - Submillimeter Array Filler Program \hfill Mar. 2019 \\
\textit{Disks around RV Tauri Variable Stars: U Mon in High Resolution} \\ Role: \textbf{Co-I}, PI: Rodolfo Montez Jr.

\item SAO - Submillimeter Array Filler Program \hfill Oct. 2018 \\
\textit{Disks around RV Tauri Variable Stars: U~Mon (0.8 mm)} \\ Role: \textbf{Co-I}, PI: Rodolfo Montez Jr.

% \item \textbf{National Optical Astronomy Observatory - Center for High Angular Resolution Astronomy - September 2018}\\
% Co-I: Declined, Diving into the close stellar environment of the magnetic
% RV Tauri star U~Mon, \\ 
% Observing Period: 16 May 2017 - 15 Nov. 2018

% \item \textbf{Smithsonian Astrophysical Observatory - Submillimeter Array - September 2018} \\
% PI: Declined, Hunting for Disks Around Pulsating Stars: The Case for RV Tau Variables, Observing Period: 16 Nov. 2018 - 15 May 2019

% \item \textbf{Palomar Observatory - April 2018} \\
% Co-I: \textbf{Declined}, Rapidly-Rotating X-ray Bright Stars: Are Stellar Mergers Common? \\ Observing Period: Aug. - Oct. 2018; PI: Dawn Gelino

% \item \textbf{Chandra X-ray Observatory - March 2018} \\
% Co-I: Declined, Resolving Intermediate-stage Feedback: A Complete View of M 51b's X-ray Morphology, Cycle 20 

%\item \textbf{Smithsonian Astrophysical Observatory - Submillimeter Array - March 2018} \\
%PI: Declined, Disks around RV Tauri Variable Stars, Observing Period: 16 May 2017 - 15 Nov. 2018

\item SAO - Submillimeter Array Filler Program \hfill Oct. 2018 \\
\textit{Disks around RV Tauri Variable Stars: U~Mon} \\ Role: \textbf{Co-I}, PI: Rodolfo Montez Jr.

%\item \textbf{Smithsonian Astrophysical Observatory - Submillimeter Array - September 2017} \\
%PI: Declined, Disks around RV Tauri Variable Stars, Observing Period: 16 Nov. 2017 - 15 May 2018

% \item \textbf{K2 Guest Observer Program - March 2016} \\
% Co-I: Declined, Understanding the Mysterious RV Tau Phenomenon with K2, PI: Keivan Stassun

\end{etaremune}

% \end{enumerate}

% \bigskip
% \newpage
\noindent
{\bf Oral Presentations} \\
\vspace{-10mm}
\begin{center}
\rule{\textwidth}{0.2mm}
\end{center}
\vspace{-3mm}
\noindent
% \begin{enumerate}[\bfseries 1.] 
\begin{etaremune}
\renewcommand\labelenumi{\bfseries\theenumi .}
% \textit{Upcoming
\item A Journey of Gratitude and Growth: The Impact of the Fisk-Vanderbilt Master's-to-PhD Bridge Program on My Path to the PhD; Special Session: Celebrating Bridge and Support Programs for Astronomy and Astrophysics Students, American Astronomical Society, AAS 245th Meeting, Washington, DC, Jan. 2025.

\item From Radio to X-ray: Tracing the flares of YZ Canis Minoris Across the Spectrum; American Astronomical Society, AAS 245th Meeting, Washington, DC, Jan. 2025.

\item From Radio to X-ray: Tracing the flares of YZ Canis Minoris Across the Spectrum; National Society of Black Physicists Conference; Houston, TX, Nov. 2024.

\item \textbf{Invited Talk}: Updates on our Multiwavelength Observations Campaign of Active M Dwarf Stars with TESS, Swift, and NICER; NASA JPL Virtual Exoplanet Lecture Series; Virtual Meeting, Pasadena, CA, Mar. 2024. 

\item Exploring Flares in Simultaneous Multiwavelength Observations of Active M Dwarf Stars; NASA/CRESST II Retreat; University of Maryland, College Park, MD, Mar. 2024. 

\item Is the Flaring Star Wolf 359 a Good Exoplanet Host?; American Astronomical Society, AAS 243rd Meeting. New Orleans, LA, Jan. 2024.

\item \textbf{Invited Talk}: Exploring Flares in Simultaneous Multiwavelength Observations of Active M Dwarf Stars; Penn State Center for Exoplanets and Habitable Worlds; State College, PA, Nov. 2023.

\item \textbf{Invited Talk}: Exploring Flares in Simultaneous Multiwavelength Observations of Active M Dwarf Stars; National Society of Black Physicists Conference; Knoxville, TN, Nov. 2023.

\item Exploring Flares in Simultaneous Multiwavelength Observations of Active M Dwarf Stars; \href{https://itc.cfa.harvard.edu/event/itc-luncheon-112}{Institute for Theory and Computation Luncheon}; Harvard--Smithsonian Center for Astrophysics, Cambridge, MA, Nov. 2023.

\item \textbf{Invited Talk}: My Research on RV Tauri Stars and M Dwarf Flares; Tecnol\'{o}gico de Monterrey Campus Sonora Norte, Hermosillo, Sonora, M\'{e}xico, Oct. 2023.

\item \textbf{Invited Talk}: Exploring RV Tauri Stars: A Study of DF Cygni and U Monocerotis; Mid-Hudson Astronomical Association, Virtual Meeting, SUNY-New Paltz, New Paltz, NY, Sep. 2023.

\item Exploring Flares in Simultaneous Multiwavelength Observations of Wolf 359; Early Career Scientist Forum 2023, Sciences and Exploration Directorate 600, Goddard Space Flight Center, Greenbelt, MD, Sep. 2023.

\item \textbf{Invited Talk}: Stellar Evolution at the Crossroads: Resolving the Nature of RV Tauri Stars using Multiwavelength Observations from Kepler, XMM-Newton, DASCH, AAVSO, and the SMA; Earth and Planets Laboratory Astronomy Seminar, Carnegie Institution for Science; Washington, DC, May 2023.

\item \textbf{Invited Talk}: Simultaneous Multiwavelength TESS, Swift, and NICER Observations of Highly Active M Dwarf stars; NASA JPL Exoplanets Journal Club; Virtual Meeting, Pasadena, CA, Nov. 2022.

\item Simultaneous Multiwavelength Observations of the Highly Active M Dwarf YZ CMi; SACNAS, San Juan, Puerto Rico, Oct. 2022.

\item \href{https://meetings.aps.org/Meeting/APR22/Session/L05.1}{The APS Bridge Program and my journey to the PhD}; Session L05: Experiences from the APS Bridge Program, APS April Meeting 2022, New York, NY, Apr. 2022.

\item \href{https://asd.gsfc.nasa.gov/conferences/UVsymposium2022/agenda/}{Multiwavelength observations of Highly Active M Dwarf stars using Swift, TESS, and NICER}; Ultraviolet Science at Goddard, NASA Goddard Space Flight Center, Greenbelt, MD, Apr. 2022.

\item \href{https://twitter.com/UTSA_PhyAst/status/1508503792261779464?s=20&t=dVQ_0wPfIXmUPicaAi6VLg}{Simultaneous Multiwavelength Swift, TESS, and NICER Observations of Highly Active M Dwarf Stars}; Invited Talk Colloquium, Virtual Meeting, UTSA Department of Physics \& Astronomy, San Antonio, TX, Apr. 2022. 

\item Resolving the Nature of RV Tauri Variable Systems Using Unprecedented Observations From the Kepler and XMM-Newton Space Telescopes; AAS Dissertation Talk, Stars IV Oral Session, AAS 238th Meeting. Virtual, Jun. 2021.

\item Stellar Evolution at the Crossroads: Resolving the Nature of RV Tauri Variable Systems Using Unprecedented Observations from Kepler and XMM-Newton Space Telescopes; Invited Talk Colloquium, Virtual Meeting, Howard University, Washington, DC, Mar. 2021.

\item Observations of Disks Around RV Tauri Variable Stars with the Submillimeter Array; SACNAS Conference, the Henry B. Gonz\'alez Convention Center, San Antonio, TX, Oct. 2018.

\item \href{https://physics.columbian.gwu.edu/sites/g/files/zaxdzs1976/f/downloads/ACDC2017_Agenda_0.pdf}{Evidence for Disk Obscuration in Kepler Observations of the RV Tau Star DF Cyg}; \nth{5} Annual Astrophysics Conference - DC, The George Washington University, Washington DC, Jul. 2017.

\item \href{https://as.vanderbilt.edu/astronomy/2016/08/journal-club-fall-2016/}{Kepler's Unprecedented High Temporal Precision Observations of the RV Tauri variable DF Cygni}, Astronomy Journal Club, Vanderbilt University, Nashville, TN, Sep. 2016.

\item \href{http://adsabs.harvard.edu/abs/2015AAS...22522708V}{M51 and the Effect of the Arm Resonance and Interaction on Diffuse X-ray Emission}; American Astronomical Society, AAS 225th Meeting. Seattle, WA, Jan. 2015.

\item \href{http://www.utsa.edu/today/2014/04/celestiallights4.html}{Cosmic Background Radiation and the Evolution of the Universe}; UTSA’s Friday Nights, Celestial Lights Series. San Antonio, TX, Apr. 18, 2014.

\item Diffuse X-ray \& Multi-wavelength Emission in Nearby, Face-on Spiral Galaxies; San Antonio, Texas Astrophysics Researchers Meeting. San Antonio, TX, Mar. 22, 2013.

\item “Diffuse X-ray \& Multi-wavelength Emission in Nearby, Face-on Spiral Galaxies” and poster presentation “Diffuse X-ray Emission and Star Clusters in Nearby, Face-on Spiral;” UTSA College of Science, COS, Research Conference. San Antonio, TX, Oct. 5, 2012.

\end{etaremune}
% \end{enumerate}

% \bigskip
% \newpage


\noindent
{\bf Poster Presentations} \\
\vspace{-10mm}
\begin{center}
\rule{\textwidth}{0.2mm}
\end{center}
\vspace{-3mm}
\noindent
% \begin{enumerate}[\bfseries 1.] 
\begin{etaremune}
\renewcommand\labelenumi{\bfseries\theenumi .}
% \item \textit{Upcoming}: 

\item From Radio to X-ray: Tracing the flares of YZ Canis Minoris Across the Spectrum; American Geophysical Union, AGU24 Annual Meeting, Washington, DC, Dec. 2024.

\item \href{https://coolstars22.github.io/docs/CS22_Abstract_booklet.pdf}{From Radio to X-ray: Tracing YZ CMi's Flares Across the
Spectrum}; Cool Stars 22, San Diego, California, Jun. 24-28, 2024.

\item Is the Flaring Star Wolf 359 a Good Exoplanet Host?; 2023 UMD Postdoctoral Research Symposium, University of Maryland, College Park, MD, Sep. 2023.

\item \href{http://www.astro.up.pt/~sousasag/toe2023/Abstract_book_Posters.pdf}{Is the Flaring Star Wolf 359 a Good Exoplanet Host?}; Towards Other Earths III: The Planet-Star connection, Porto, Portugal, Jul. 17-21, 2023.

\item Simultaneous Multiwavelength Swift, TESS, and NICER Observations of Highly Active M Dwarf Stars; 2022 UMD Postdoctoral Research Symposium, University of Maryland, College Park, MD, Sep. 2022.

\item \href{https://coolstars21.github.io/docs/CS21booklet.pdf}{Simultaneous Multiwavelength Swift, TESS, and NICER Observations of Highly Active M Dwarf Stars}; Cool Stars 21, Toulouse, France, Jul. 4-9, 2022.

\item \href{https://zenodo.org/record/5142127#.YmbZsJPMJCV}{Simultaneous Multiwavelength Observations of the Highly Active M Dwarf YZ CMi}; TESS Science Conference II, Virtual Meeting, Aug. 2021.

\item \href{https://ui.adsabs.harvard.edu/abs/2020AAS...23510607V/abstract}{Multiwavelength Observations of the RV Tau Variable U Monocerotis}; American Astronomical Society, AAS 235th Meeting.  Honolulu, HI, Jan.  2020.

\item \href{http://www.stsci.edu/~ofox/posters2019/}{X-ray \& Submillimeter Observations of the Pulsating RV Tau Variable U Mon}; 2019 Spring Symposium - The Deaths and Afterlives of Stars; Space Telescope Science Institute; Apr. 2019.

\item \href{https://sites.google.com/view/ncad7-at-jhu/program?authuser=0}{Submillimeter Array Observations of the Large Disk Around RV Tau Star U~Mon}; 7th National Capital Area Disks Meeting; Johns Hopkins University / Space Telescope Science Institute; 26 Sep., 2018.

\item \href{http://coolstars20.cfa.harvard.edu/abstracts.html#posters}{Observations of Disks Around RV Tau Variables with the Submillimeter Array}; Cool Stars Conference; Boston University, Boston, MA, Jul. 2018.

\item Evidence for Disk Obscuration in Kepler Observations of the Pulsating RV Tau Variable DF Cyg; Kepler-K2 Science Conference; NASA Ames, Mountain View, CA, Jun. 2017.

\item \href{http://adsabs.harvard.edu/abs/2017AAS...22915210V}{Evidence for Binarity in Kepler Observations of the Pulsating RV Tau Variable DF Cygni}; American  Astronomical  Society,  AAS  229th  Meeting. Grapevine, TX, Jan. 2017.

\item \href{https://physics.columbian.gwu.edu/sites/g/files/zaxdzs1976/f/downloads/Astro2016_Conf_program_final.pdf}{Stellar  Evolution at the Crossroads: A Closer Look at the RV Tauri Variable Star DF Cygni}; \nth{4} Annual DC/MD/VA Astrophysics Meeting, The George Washington University, Washington DC, Jun. 2016.

\item Stellar Evolution at the Crossroads: A Closer Look at the RV Tauri Variable Star DF Cygni; Fisk University Symposium, Nashville, TN, Aug. 2015. 

\item Stellar Evolution at the Crossroads: A Closer Look at the RV Tauri Variable Star DF Cygni; Fisk-Vanderbilt Masters-to-PhD Bridge Program, Research Celebration Day. Nashville, TN, Aug. 2015. 

\item \href{http://adsabs.harvard.edu/abs/2014xvge.confE..38S}{Diffuse Hot Gas in M51}; Chandra’s X-ray View of Galaxy Ecosystems; 
% Schlegel, Eric M.; \textbf{Vega,~L.~D.}; Moore, M.; 
Boston, MA. Jul. 2014.

\item \href{http://adsabs.harvard.edu/abs/2014AAS...22345311S}{Diffuse Emission in Nearby, Face-on Spiral Galaxies}; AAS 223rd Meeting;
% Schlegel, Eric M.; \textbf{Vega, L. D.}; Moore, M.; 
Washington, DC, Jan. 5-9, 2014.

\item \href{http://adsabs.harvard.edu/abs/2012AAS...21934630V}{Diffuse X-ray Emission and Star Clusters in Nearby, Face-on Spiral Galaxies}; AAS 219th Meeting. Austin, TX, Jan. 7-12 2012.

\item \href{http://adsabs.harvard.edu/abs/2011AAS...21733813M}{X-ray Eclipses in the 2010 Outburst of the Recurrent Nova U Sco?} \& \href{http://adsabs.harvard.edu/abs/2011AAS...21733812M}{Observations of the 2010 January Outburst of the Recurrent Nova U Sco using NASA's Swift}; AAS 117th Meeting;
% McMaster, Laura; \textbf{Vega, L.}, et al.; 
Seattle, WA, Jan. 9-13 2011. 


% \item \textbf{Vega, Laura D.}; Branum, A.; Burns, C.; Hoffman, N.; Kim, H.; McMaster, L.; Moore, M.; Schlegel, E. M.; “Diffuse X-ray Emission and Star Clusters in Nearby, Face-on Spiral;” South Central Conference for Undergraduate Women in Physics, SCUWiP. Austin, TX, Jan. 18-20 2013. 

%\item McMaster, Laura; \textbf{Vega, L.}; Moore, M.; Engelhardt, S.; Schlegel E.; Pagnotta, A.; “Observations of the 2010 January Outburst of the Recurrent Nova U Sco using NASA's Swift” \& “X-ray Eclipses in the 2010 Outburst of the Recurrent Nova U Sco?” SCUWiP. College Station, TX, Jan. 13-15 2012.

% \item \textbf{Vega, Laura D.}; Marzal, P.; Garcia-Martinez, I.; Cruz-Gonzalez, N.; Larios-Rodriguez, E.; Yacaman, M. J.;  “Molybdenum Sulfide at a Nanoscale: Structure and Catalytic Analyses;” UTSA College of Science, COS, Research Conference. San Antonio, TX, Oct. 5, 2012.

%\item McMaster, Laura; \textbf{Vega, L.}; Moore, M.; Engelhardt, S.; Schlegel E.; Pagnotta, A.; “X-ray Eclipses in the 2010 Outburst of the Recurrent Nova U Sco?” AND “Observations of the 2010 January Outburst of the Recurrent Nova U Sco using NASA's Swift;” UTSA COS Research Conference. San Antonio, TX, Sep. 20, 2011.

\item \href{http://adsabs.harvard.edu/abs/2010APS..TSFFP1019V}{“X-ray Eclipses in the 2010 Outburst of the Recurrent Nova U Sco?”} \&  \href{http://adsabs.harvard.edu/abs/2010APS..TSFFP1013M}{“Observations of the 2010 January Outburst of the Recurrent Nova U Scoprius using NASA's Swift”}; Joint Fall 2010 Meeting of the Texas Sections of the APS, AAPT, Zone 13 of SPS and the National Society of Hispanic Physicists, San Antonio, TX, Oct. 21-23, 2010.

\end{etaremune}

\bigskip
% % \newpage
% \noindent
% {\bf Interviews} \\
% \vspace{-10mm}
% \begin{center}
% \rule{\textwidth}{0.2mm}
% \end{center}
% \vspace{-3mm}
% \begin{itemize}
\noindent
{\bf Interviews} \\
\vspace{-10mm}
\begin{center}
\rule{\textwidth}{0.2mm}
\end{center}
\vspace{-3mm}
\begin{itemize}
    \setlength{\itemsep}{0pt} % Removes space between items
    \item \textbf{\href{https://fb.watch/nBDyaBPEFK/?mibextid=Nif5oz}{NASA Space Apps Challenge Hermosillo}} | Noticias Telemax Sonora, MX \hfill \\ Oct. 2023
    \item \textbf{\href{https://cresst2.umd.edu/scientistofmonth/February2023-Vega_Laura.pdf}{Scientist of the Month}} | NASA-Goddard CRESST II \hfill Feb. 2023
    \item \textbf{\href{https://www.aps.org/apsnews/2022/12/bridge-program-nasa}{First a Bridge Program Graduate, Now a NASA Astrophysicist}} | APS News \hfill Jan. 2023
    \item \textbf{\href{https://science.gsfc.nasa.gov/600/ECSS/Laura-Vega.html}{Early Career Spotlight}} | NASA Sciences and Exploration Directorate \hfill Oct. 2022
    \item \textbf{\href{https://animalpolitico.com/tendencias/ciencia-tecnologia/mexicanos-en-la-nasa-entrevista-laura-vega-fernando-mier-hicks}{Ver Oportunidades En El Fracaso Y Persistir}} | Animal MX Journal \hfill Sep. 2022
    \item \textbf{\href{https://sciences.utsa.edu/spotlights/alumni/2020/laura-vega.html}{Alumni Spotlight}} | UTSA College of Sciences \hfill Dec. 2020
\end{itemize}

% \bigskip

\newpage
\noindent
{\bf Professional Memberships/Affiliations} \\
\vspace{-10mm}
\begin{center}
\rule{\textwidth}{0.2mm}
\end{center}
\vspace{-3mm}
{ 

\small 
\noindent 
American Geophysical Union (AGU) \hfill 2024 - Present \\
National Society of Black Physicists (NSBP) \hfill 2023 - Present \\
Latinx Employee Association (LEA) at UMD \hfill 2021 - Present \\
Vanderbilt Association of Hispanic and Latinx Alumni \hfill 2021 - Present \\
Sigma Xi \text{\textbar} The Scientific Research Society ($\Sigma \Xi$) \hfill 2015 - Present \\
National Society of Hispanic Physicists (NSHP) \hfill 2013 - Present \\
Toastmasters International \hfill 2013 - Present \\
Society for Advancement of Chicanos \& Native Americans in Science (SACNAS) \hfill 2013 - Present  \\
American Physical Society (APS) \hfill 2011 - Present \\
American Astronomical Society (AAS) \hfill 2011 - Present \\
Sigma Pi Sigma \text{\textbar} The Physics Honor Society ($\Sigma \Pi \Sigma$) \hfill 2011 - Present \\
LeaderShape Institute Alumna \& Pebble Club \hfill 2011 - Present \\
UTSA Alumni Association \hfill 2013 - 2014 \\
San Antonio Astronomical Association \hfill 2013 - 2014 \\
UTSA San Antonio Texas Astrophysics Researchers (STAR) \hfill 2011 - 2014 \\
Society of Physics Students (SPS) \hfill 2010 - 2013 \\
National Society of Collegiate Scholars (NSCS) \hfill 2008 - 2013 \\
}
% \bigskip

\newpage
\noindent
{\bf References} \\
\vspace{-10mm}
\begin{center}
\rule{\textwidth}{0.2mm}
\end{center}
\vspace{-3mm}

\noindent
\href{http://astro.phy.vanderbilt.edu/~stassuk/}{\textbf{Keivan G. Stassun}}, Stevenson Professor of Astrophysics,  College of Arts \& Science \\
Professor of Computer Science, Vanderbilt School of Engineering \\
Professor of Management,     Owen Graduate School of Management \\
Director,  Frist Center for Autism and Innovation at Vanderbilt \\
Vanderbilt University, Nashville, TN \\
% (615) 322-2828 \\
keivan.stassun@vanderbilt.edu \\
\textit{Dissertation Advisor \& Chair} \\

\noindent
\href{http://www.rudyphd.com/}{\textbf{Rodolfo Montez, Jr.}}, Astrophysicist \\
Smithsonian Astronomical Observatory, Cambridge, MA \\
% (617) 496-7565 \\
rodolfo.montez@cfa.harvard.edu \\
\textit{Graduate Research Mentor, Dissertation Committee Member} \\

\noindent
\href{https://sciences.utsa.edu/faculty/profiles/schlegel-eric.html}{\textbf{Eric M. Schlegel}}, Vaughan Family Professor of Astronomy \\
The University of Texas at San Antonio, San Antonio, TX \\
% (210) 458-6425 \\
eric.schlegel@utsa.edu \\
\textit{Dissertation Committee Member, Former Undergraduate Research Advisor} \\

\noindent
\href{https://science.gsfc.nasa.gov/sed/bio/patricia.t.boyd}{\textbf{Patricia T. Boyd}}, Director, NASA MOSAICS (formerly SMD Bridge) Program \\
% Former Chief, Exoplanets and Stellar Astrophysics Laboratory at NASA Goddard \\
NASA Headquarters, Washington, DC \\
% (301) 286-2550 \\
patricia.t.boyd@nasa.gov \\
\textit{Dissertation Committee Member, Former (Graduate and Postdoctoral) Advisor} \\

\noindent
\href{https://science.gsfc.nasa.gov/sci/bio/elisa.quintana}{\textbf{Elisa V. Quintana}}, Research Astrophysicist \\ NASA Goddard Space Flight Center, Greenbelt, MD \\
elisa.quintana@nasa.gov \\
\textit{Postdoctoral Co-Advisor} \\

\noindent
\href{https://science.gsfc.nasa.gov/sed/bio/allison.a.youngblood}{\textbf{Allison A. Youngblood}}, Research Astrophysicist \\ NASA Goddard Space Flight Center, Greenbelt, MD \\
allison.a.youngblood@nasa.gov \\
\textit{Postdoctoral Co-Advisor} \\

\noindent
\href{https://www.bucknell.edu/fac-staff/jackie-villadsen}{\textbf{Jackie Villadsen}}, Assistant Professor of Physics \& Astronomy \\ Bucknell University, Lewisburg, PA \\
jrv012@bucknell.edu \\
\textit{Postdoctoral Research Mentor} \\


\end{document}
