\documentclass[letter,11pt]{article}
\usepackage{graphics,graphicx,amssymb,amsmath,enumerate}
\usepackage{fancyhdr}
\usepackage[nodayofweek]{datetime}
\usepackage{lastpage}
\usepackage{etaremune}

\usepackage{hyperref}
\hypersetup{
    colorlinks=true,
    linkcolor=blue,
    filecolor=magenta,      
    urlcolor=blue,
}
 
\urlstyle{same}

\usepackage[super]{nth}

\setlength{\textwidth}{6.5in} 
\setlength{\textheight}{9.4in}
\setlength{\topmargin}{-0.7in} 
\setlength{\oddsidemargin}{0in}
\setlength{\evensidemargin}{0in} 

\newdateformat{monthyeardate}{\monthname[\THEMONTH] \THEYEAR}

\begin{document}
%\pagenumbering{gobble}% Remove page numbers (and reset to 1)
\pagestyle{fancy}
\fancyhf{}
\renewcommand{\headrulewidth}{0pt}

\rhead{\footnotesize{\thepage\ of \pageref{LastPage}}}
\lfoot{\footnotesize{L.D. Vega}}
\rfoot{\footnotesize{Updated \monthyeardate\today}} 

\begin{center} 
\bfseries{
\LARGE \uppercase{Laura D. Vega} \\
\large \uppercase{Curriculum Vitae}}
\end{center}

\noindent
\begin{center}{\bf Astrophysicist \(|\) Heising-Simons Foundation Astrophysics Postdoctoral Fellow} \\
% Vanderbilt University - Department of Physics \& Astronomy}
\end{center}

\noindent Department of Astronomy \hfill \textit{Email:} ldvega@umd.edu \\
\noindent University of Maryland, College Park \hfill \href{https://www.astro.umd.edu/people/directory.html#Postdocs\%20and\%20Faculty\%20Assistants}{UMD Astro Directory}  \\
4296 Stadium Drive, College Park, MD 20742
% 6301 Stevenson Center, Nashville, TN 37235 \hfill Citizenship: United States of America \\
%  NASA Goddard Space Flight Center \hspace{7.0cm} \url{http://my.vanderbilt.edu/lauradvega}
\vskip 0.1in
\noindent Exoplanets and Stellar Astrophysics Laboratory \hfill \textit{Email:} laura.d.vega@nasa.gov
\\ 
NASA Goddard Space Flight Center \hfill
\url{https://science.gsfc.nasa.gov/sed/bio/laura.d.vega} \\
Mail Code 667, 8800 Greenbelt Rd, Greenbelt, MD 20771 \\

\noindent Center for Research and Exploration in Space Science and Technology II (CRESST II) \\ NASA/GSFC, Greenbelt, MD 20771 \\

%6301 Stevenson Center \hspace{4.35cm} 
% Nashville, TN 37235 \hspace{4.72cm}
% \bigskip

\noindent
{\bf Research Interests} \\
\vspace{-10mm}
\begin{center}
\rule{\textwidth}{0.2mm}
\end{center}
\vspace{-3mm}
\noindent
% Astrophysics Ph.D. candidate at Vanderbilt University stationed at NASA-GSFC. 
My research interests include multi-wavelength astronomy, late-stage stellar evolution, post-AGB binaries, circumbinary disks, and X-ray/UV flares in M-dwarf stars.

\bigskip

\noindent
{\bf Education} \\
\vspace{-10mm}
\begin{center}
\rule{\textwidth}{0.2mm}
\end{center}
\vspace{-3mm}
\noindent
Astrophysics Ph.D., 2021 -- Vanderbilt University \hfill \textit{Advisor:} Keivan Stassun \\
Physics M.A., 2017 -- Fisk University \hfill \enspace \qquad \textit{Advisor:} Keivan Stassun \\
Physics B.Sc., 2013 -- The University of Texas at San Antonio (UTSA) \hfill \textit{Advisor:} Eric Schlegel \\

\noindent
{\bf Publications} ([Number] = Number of citations for refereed papers as of CV's date-of-update) \\
\vspace{-10mm}
\begin{center}
\rule{\textwidth}{0.2mm}
\end{center}
\vspace{-3mm}
\noindent
% \begin{enumerate}[\bfseries 1.] 
\begin{etaremune}
\renewcommand\labelenumi{\bfseries\theenumi .}

\item {\bf Vega L.D. et al. 2023, In Prep.}; {\it Simultaneous TESS and Swift Observations of Wolf~359}.

\item {\bf Vega L.D. et al. 2023, In Prep.}; {\it Multiwavelength Observations of the YZ~CMi System}.

\item {Wittrock, Justin M. et al. 2023, Submitted.};\href{https://arxiv.org/abs/2302.04922}{\it Validating AU Microscopii d with Transit Timing Variations}

\item {Chavali, S. et al. 2022};\href{https://ui.adsabs.harvard.edu/abs/2022RNAAS...6..201C/abstract}{\it A Pilot Survey of an M Dwarf Flare Star with Swift’s UV Grism}, Research Notes of the AAS, 6, 201.

\item {Wittrock, Justin M. et al. 2022}; \href{https://ui.adsabs.harvard.edu/abs/2022AJ....164...27W/abstract}{\it Transit Timing Variations for AU Microscopii b and c}, The Astronomical Journal, 164, 27. [4]

\item Gilbert E.A. et al. 2022; \href{https://ui.adsabs.harvard.edu/abs/2022AJ....163..147G/abstract}{\it Flares, Rotation, and Planets of the AU Mic System from TESS Observations}, The Astronomical Journal, 163, 147. [25]

\item {Paudel R.R. et al. 2021}; 
\href{https://ui.adsabs.harvard.edu/abs/2021ApJ...922...31P/abstract}{\it Simultaneous Multiwavelength Flare Observations of EV Lacertae}, The Astrophysical Journal, 922, 31. [10]

\item {\bf Vega L.D. et al. 2021}; \href{https://ui.adsabs.harvard.edu/abs/2021ApJ...909..138V/abstract}{\it Multiwavelength Observations of the RV Tauri Variable System U Monocerotis: Long-Term Variability Phenomena Can Be Explained by Binary Interactions With a Circumbinary Disk}, The Astrophysical Journal, 909, 138. (\href{https://www.nasa.gov/feature/goddard/2021/scientists-sketch-aged-star-system-using-over-a-century-of-observations/}{\bf NASA Feature}). [3]

\item Gilbert E.A. et al. 2020; \href{https://ui.adsabs.harvard.edu/abs/2020AJ....160..116G/abstract}{\it The First Habitable Zone Earth-sized Planet from TESS. I: Validation of the TOI-700 System}, The Astronomical Journal, 160, 116.
(\href{https://www.nasa.gov/feature/goddard/2020/nasa-planet-hunter-finds-its-1st-earth-size-habitable-zone-world}{\bf NASA Feature}). [67]

\item Kostov, V.B. et al. 2019; \href{https://ui.adsabs.harvard.edu/abs/2019AJ....158...32K/abstract}{{\it The L 98-59 System: Three Transiting, Terrestrial-size Planets Orbiting a Nearby M Dwarf}}, The Astronomical Journal, 158, 32.
(\href{https://www.nasa.gov/feature/goddard/2019/nasa-s-tess-mission-finds-its-smallest-planet-yet}{\bf NASA Feature}). [78]

\item {\bf Vega, L.D. et al. 2017}; \href{https://ui.adsabs.harvard.edu/abs/2017ApJ...839...48V/abstract}{{\it Evidence for Binarity and Possible Disk Obscuration in Kepler Observations of the Pulsating RV Tau Variable DF~Cygni}}, The Astrophysical Journal, 839, 48. [10]

\item Schlegel, E.M. et al. 2016; \href{https://ui.adsabs.harvard.edu/abs/2016ApJ...823...75S/abstract}{{\it NGC 5195 In M51: Feedback ‘Burps’ After a Massive Meal?}}, The Astrophysical Journal, 823, 75. (\href{https://www.nasa.gov/mission_pages/chandra/nasa-s-chandra-finds-supermassive-black-hole-burping-nearby.html}{\bf NASA Feature}). [9]

\end{etaremune}
% \end{enumerate}

% \bigskip
% \newpage

\noindent
{\bf Relevant Skills} \\
\vspace{-10mm}
\begin{center}
\rule{\textwidth}{0.2mm}
\end{center}
\vspace{-3mm}
\noindent
\textbf{Languages}: English; Spanish\\
\textbf{Computer (varying levels of proficiency)}: Unix; Python; LaTeX; Lightkurve; HEASARC: FTOOLS, XSPEC; SAOImage DS9 \\
%X-ray Spectral Fitting Package (XSPEC), Chandra Interactive Analysis of Observations (CIAO) \\
\textbf{Data experience (varying levels of proficiency)}: X-ray/Ultraviolet (NICER, {\it Swift}); Optical ({\it Kepler}, TESS, AAVSO, DASCH); and Radio (SMA) \\
% \indent Space Telescopes: \\ X-ray/UV: Chandra, Swift, Kepler, and the Transiting Exoplanet Survey Satellite. \\ \indent Ground-based: \\ American Astronomical Variable Stars Observers, Digital Access to a Sky Century @ Harvard. \\ \indent Interferometry: \\ The Smithsonian Sub-millimeter Array (Radio) \\

\noindent
{\bf Fellowships/Grants} \\
\vspace{-10mm}
\begin{center}
\rule{\textwidth}{0.2mm}
\end{center}
\vspace{-3mm}
\noindent
NASA TESS Cycle 4 Guest Investigator Program \hfill 2021 - 2022 \\
NASA TESS Cycle 3 Guest Investigator Program \hfill 2020 - 2021 \\
NASA Minority University Research Education Project (MUREP)/Advanced STEM Training and Research (ASTAR): Harriett G. Jenkins Pre-doctoral Fellowship \hfill 2015 - 2020 \\
Smithsonian Latino Center - LIP (Research at the Center for Astrophysics and Outreach at National Air \& Space Museum) \hfill 2017, 2018, 2019 \\
Vaughan Family Grant (AAS Conference Travel (UTSA)) \hfill 2011, 2012, 2014 \\
National Institute on Minority Health and Health Disparities – Nanotechnology and Human Health Core Grant (2012 Summer Undergraduate Nanoparticles Research at UTSA) \hfill 2012 \\
Bill \& Alicia Hoover Grant (For 2010 Summer Research at UTSA) \hfill 2010 \\

% \bigskip

\noindent
{\bf Honors and Awards} \\
\vspace{-10mm}
\begin{center}
\rule{\textwidth}{0.2mm}
\end{center}
\vspace{-3mm}
\noindent
Most Outstanding Student Publication - Vanderbilt Physics \& Astronomy Department \hfill 2018 \\
NASA Office of Education's MUREP: Harriett G. Jenkins Pre-doctoral Fellow \hfill 2015 \\
Best Poster in Physics - Fisk University Science Symposium \hfill 2016 \\
Toastmasters International - Competent Leader \hfill 2015 \\
Toastmasters International - Competent Communicator \hfill 2015 \\
The Sam Madrid Jr. Endowed Scholarship - UTSA \hfill 2012 \\
The Dr. B. Thyagarajan Endowed Scholarship - UTSA \hfill 2009 \\
Dean’s List - UTSA \hfill 2007, 2008 \\
Honor Roll - UTSA \hfill 2009, 2010, 2011 \\


\noindent
{\bf Observation Proposals} \\
\vspace{-10mm}
\begin{center}
\rule{\textwidth}{0.2mm}
\end{center}
\vspace{-3mm}
\noindent
% \begin{enumerate}[\bfseries 1.] 
\begin{etaremune}
\renewcommand\labelenumi{\bfseries\theenumi .}

\item ALMA - Cycle 9 - April 2022 \\
Co-I: ACCEPTED; The Origin and Impact of Flares in M Dwarf Systems; PI: Meredith MacGregor

\item TESS - Cycle 5 - January 2022 \\
Co-I: ACCEPTED; Multiwavelength TESS-Swift-NICER Observations of Pulsations in Flares on Solar-Type Stars; PI: Teresa Monsue

\item Swift - Cycle 18 - September 2021 \\
Co-I: ACCEPTED: A Swift and Alma View of the Origin and Impact of M-Dwarf Flares; PI: Ward Howard

\item TESS - Cycle 4 - January 2021 \\
Co-I: ACCEPTED; Using Tess 20-S Cadence Data To Study Flares On M Dwarfs; PI: Rishi Paudel

\item NICER - Cycle 3 - November 2020 \\
Co-I: ACCEPTED; A Study of M Dwarf Flares Using Simultaneous High Cadence Multi-wavelength Data; PI: Rishi Paudel

\item XMM-Newton - Announcement of Opportunity 20 - October 2020 \\
Co-I: ACCEPTED; Pilot Study of RV Tau Variables: A new class of X-ray emitting stars? PI: Rodolfo Montez Jr.

% \item Swift - Cycle 17 - September 2020 \\
% Co-I: SUBMITTED; Using High Cadence Simultaneous Multi-wavelength Data to Study Flares on Highly Active M Dwarfs; PI: Teresa Monsue

\item Hubble Space Telescope Cycle 28 - October 2020 \\
Co-I: ACCEPTED; Confirming a Tentative Detection of an Atmosphere Around a Potentially Rocky Planet; PI: Thomas Barclay

\item Swift - Target of Opportunity - July 2020 \\
\textbf{PI: ACCEPTED}; 20 kilo-second UVOT observations of the AU~Mic system

\item Joint TESS-Swift - Cycle 3 - January 2020 \\
\textbf{PI: ACCEPTED}; Exploring the Star-Planet Connection via Simultaneous TESS and Swift Observations of Highly Active M Dwarfs

\item TESS Cycle 3 - January 2020 \\
Co-I: ACCEPTED; Discovering Circumbinary Planets With Tess; PI: Veselin Kostov

\item NICER - Cycle 2 - November 2019 \\
Co-I: ACCEPTED; Multiwavelength Observations of Highly Active M Dwarfs; PI: Rishi Paudel

% \item \textbf{ALMA - Cycle 7 - April 2019} \\
% Co-I: \textbf{SUBMITTED}; The Origin and Impact of Flares in M Dwarf Systems- Monitoring of Wolf 359; PI: Meredith MacGregor

\item Hubble Space Telescope Cycle 27 - April 2019 \\
Co-I: ACCEPTED; Searching for Secondary Atmospheres in a System of Benchmark Worlds; PI: Thomas Barclay

\item NOAO - Center for High Angular Resolution Astronomy - April 2019\\
Co-I: ACCEPTED; Diving into the close stellar environment of the magnetic
RV Tauri star U~Mon; PI: Laurence Sabin

% \item \textbf{Chandra X-ray Observatory - Cycle 21 - March 2019} \\
% Co-I: \textbf{SUBMITTED}; A High-Resolution Image of NGC 5195 in M51; PI: Eric Schlegel \\
% Co-I: \textbf{SUBMITTED}; A High-spatial Resolution, High-time Resolution, Sensitive Broadband X-ray Spectral Study of the Neutron Stars and Supermassive Black Holes in M51; PI: Murray Brightman

\item SAO - Submillimeter Array - March 2019 \\
 \textbf{PI: ACCEPTED}; Hunting for Disks Around Pulsating RV Tau Stars

\item SAO - Submillimeter Array (Filler Program) - March 2019 \\
Co-I: ACCEPTED; Disks around RV Tauri Variable Stars: U Mon in High Resolution; PI: Rodolfo Montez Jr.

\item SAO - Submillimeter Array (Filler Program) - October 2018 \\
Co-I: ACCEPTED; Disks around RV Tauri Variable Stars: U~Mon (0.8 mm); PI: Rodolfo Montez Jr.

% \item \textbf{National Optical Astronomy Observatory - Center for High Angular Resolution Astronomy - September 2018}\\
% Co-I: Declined, Diving into the close stellar environment of the magnetic
% RV Tauri star U~Mon, \\ 
% Observing Period: 16 May 2017 - 15 Nov. 2018

% \item \textbf{Smithsonian Astrophysical Observatory - Submillimeter Array - September 2018} \\
% PI: Declined, Hunting for Disks Around Pulsating Stars: The Case for RV Tau Variables, Observing Period: 16 Nov. 2018 - 15 May 2019

% \item \textbf{Palomar Observatory - April 2018} \\
% Co-I: \textbf{Declined}, Rapidly-Rotating X-ray Bright Stars: Are Stellar Mergers Common? \\ Observing Period: Aug. - Oct. 2018; PI: Dawn Gelino

% \item \textbf{Chandra X-ray Observatory - March 2018} \\
% Co-I: Declined, Resolving Intermediate-stage Feedback: A Complete View of M 51b's X-ray Morphology, Cycle 20 

%\item \textbf{Smithsonian Astrophysical Observatory - Submillimeter Array - March 2018} \\
%PI: Declined, Disks around RV Tauri Variable Stars, Observing Period: 16 May 2017 - 15 Nov. 2018

\item SAO - Submillimeter Array (Filler Program) - October 2018 \\
Co-I: ACCEPTED; Disks around RV Tauri Variable Stars: U~Mon; PI: Rodolfo Montez Jr.

%\item \textbf{Smithsonian Astrophysical Observatory - Submillimeter Array - September 2017} \\
%PI: Declined, Disks around RV Tauri Variable Stars, Observing Period: 16 Nov. 2017 - 15 May 2018

% \item \textbf{K2 Guest Observer Program - March 2016} \\
% Co-I: Declined, Understanding the Mysterious RV Tau Phenomenon with K2, PI: Keivan Stassun

\end{etaremune}

% \end{enumerate}

% \bigskip
\newpage
\noindent
{\bf Oral Presentations} \\
\vspace{-10mm}
\begin{center}
\rule{\textwidth}{0.2mm}
\end{center}
\vspace{-3mm}
\noindent
% \begin{enumerate}[\bfseries 1.] 
\begin{etaremune}
\renewcommand\labelenumi{\bfseries\theenumi .}

\item Simultaneous Multiwavelength TESS, Swift, and NICER Observations of Highly Active M Dwarf stars, NASA JPL Exoplanets Journal Club; Invited Talk Colloquium, Virtual Meeting, Pasadena, CA, Nov. 2022.

\item Simultaneous Multiwavelength Observations of the Highly Active M Dwarf YZ CMi, SACNAS, San Juan, Puerto Rico, Oct. 2022.

\item \href{https://meetings.aps.org/Meeting/APR22/Session/L05.1}{The APS Bridge Program and my journey to the PhD}; Session L05: Experiences from the APS Bridge Program, APS April Meeting 2022, New York, NY, Apr. 2022.

\item \href{https://asd.gsfc.nasa.gov/conferences/UVsymposium2022/agenda/}{Multiwavelength observations of Highly Active M Dwarf stars using Swift, TESS, and NICER}; Ultraviolet Science at Goddard, NASA Goddard Space Flight Center, Greenbelt, MD, Apr. 2022.

\item \href{https://twitter.com/UTSA_PhyAst/status/1508503792261779464?s=20&t=dVQ_0wPfIXmUPicaAi6VLg}{Simultaneous Multiwavelength Swift, TESS, and NICER Observations of Highly Active M Dwarf Stars}; Invited Talk Colloquium, Virtual Meeting, UTSA Department of Physics \& Astronomy, San Antonio, TX, Apr. 2022. 

\item Resolving the Nature of RV Tauri Variable Systems Using Unprecedented Observations From the Kepler and XMM-Newton Space Telescopes; AAS Dissertation Talk, Stars IV Oral Session, AAS 238th Meeting. Virtual, Jun. 2021.

\item Stellar Evolution at the Crossroads: Resolving the Nature of RV Tauri Variable Systems Using Unprecedented Observations from Kepler and XMM-Newton Space Telescopes; Invited Talk Colloquium, Virtual Meeting, Howard University, Washington, DC, Mar. 2021.

\item Astrophysics with TESS; NASA Hyperwall; American Astronomical Society, AAS 235th Meeting.  Honolulu, HI, Jan. 2020.

\item Observations of Disks Around RV Tauri Variable Stars with the Submillimeter Array; SACNAS Conference, the Henry B. Gonz\'alez Convention Center, San Antonio, TX, Oct. 2018.

\item \href{https://physics.columbian.gwu.edu/sites/g/files/zaxdzs1976/f/downloads/ACDC2017_Agenda_0.pdf}{Evidence for Disk Obscuration in Kepler Observations of the RV Tau Star DF Cyg}; \nth{5} Annual Astrophysics Conference - DC, The George Washington University, Washington DC, Jul. 2017.

\item \href{https://as.vanderbilt.edu/astronomy/2016/08/journal-club-fall-2016/}{Kepler's Unprecedented High Temporal Precision Observations of the RV Tauri variable DF Cygni}, Astronomy Journal Club, Vanderbilt University, Nashville, TN, Sep. 2016.

\item \href{http://adsabs.harvard.edu/abs/2015AAS...22522708V}{M51 and the Effect of the Arm Resonance and Interaction on Diffuse X-ray Emission}; American Astronomical Society, AAS 225th Meeting. Seattle, WA, Jan. 2015.

\item \href{http://www.utsa.edu/today/2014/04/celestiallights4.html}{Cosmic Background Radiation and the Evolution of the Universe}; UTSA’s Friday Nights, Celestial Lights Series. San Antonio, TX, April 18, 2014.

\item Diffuse X-ray \& Multi-wavelength Emission in Nearby, Face-on Spiral Galaxies; San Antonio, Texas Astrophysics Researchers Meeting. San Antonio, TX, Mar. 22, 2013.

\item “Diffuse X-ray \& Multi-wavelength Emission in Nearby, Face-on Spiral Galaxies” and poster presentation “Diffuse X-ray Emission and Star Clusters in Nearby, Face-on Spiral;” UTSA College of Science, COS, Research Conference. San Antonio, TX, Oct. 5, 2012.

\end{etaremune}
% \end{enumerate}

% \bigskip
\newpage

\noindent
{\bf Poster Presentations} \\
\vspace{-10mm}
\begin{center}
\rule{\textwidth}{0.2mm}
\end{center}
\vspace{-3mm}
\noindent
% \begin{enumerate}[\bfseries 1.] 
\begin{etaremune}
\renewcommand\labelenumi{\bfseries\theenumi .}

\item \href{https://coolstars21.github.io/docs/CS21booklet.pdf}{Simultaneous Multiwavelength Swift, TESS, and NICER Observations of Highly Active M Dwarf Stars}; Cool Stars 21, Toulouse, France, Jul. 4-9, 2022.

\item \href{https://zenodo.org/record/5142127#.YmbZsJPMJCV}{Simultaneous Multiwavelength Observations of the Highly Active M Dwarf YZ CMi}; TESS Science Conference II, Virtual Meeting, Aug. 2021.

\item \href{https://ui.adsabs.harvard.edu/abs/2020AAS...23510607V/abstract}{Multiwavelength Observations of the RV Tau Variable U Monocerotis}; American Astronomical Society, AAS 235th Meeting.  Honolulu, HI, Jan.  2020.

\item \href{http://www.stsci.edu/~ofox/posters2019/}{X-ray \& Submillimeter Observations of the Pulsating RV Tau Variable U Mon}; 2019 Spring Symposium - The Deaths and Afterlives of Stars; Space Telescope Science Institute; April 2019.

\item \href{https://sites.google.com/view/ncad7-at-jhu/program?authuser=0}{Submillimeter Array Observations of the Large Disk Around RV Tau Star U~Mon}; 7th National Capital Area Disks Meeting; Johns Hopkins University / Space Telescope Science Institute; 26 Sep., 2018.

\item \href{http://coolstars20.cfa.harvard.edu/abstracts.html#posters}{Observations of Disks Around RV Tau Variables with the Submillimeter Array}; Cool Stars Conference; Boston University, Boston, MA, July 2018.

\item Evidence for Disk Obscuration in Kepler Observations of the Pulsating RV Tau Variable DF Cyg; Kepler-K2 Science Conference; NASA Ames, Mountain View, CA, Jun. 2017.

\item \href{http://adsabs.harvard.edu/abs/2017AAS...22915210V}{Evidence for Binarity in Kepler Observations of the Pulsating RV Tau Variable DF Cygni}; American  Astronomical  Society,  AAS  229th  Meeting. Grapevine, TX, Jan. 2017.

\item \href{https://physics.columbian.gwu.edu/sites/g/files/zaxdzs1976/f/downloads/Astro2016_Conf_program_final.pdf}{Stellar  Evolution at the Crossroads: A Closer Look at the RV Tauri Variable Star DF Cygni}; \nth{4} Annual DC/MD/VA Astrophysics Meeting, The George Washington University, Washington DC, Jun. 2016.

\item Stellar Evolution at the Crossroads: A Closer Look at the RV Tauri Variable Star DF Cygni; Fisk University Symposium, Nashville, TN, Aug. 2015. 

\item Stellar Evolution at the Crossroads: A Closer Look at the RV Tauri Variable Star DF Cygni; Fisk-Vanderbilt Masters-to-PhD Bridge Program, Research Celebration Day. Nashville, TN, Aug. 2015. 

\item \href{http://adsabs.harvard.edu/abs/2014xvge.confE..38S}{Diffuse Hot Gas in M51}; Chandra’s X-ray View of Galaxy Ecosystems; Schlegel, Eric M.; \textbf{Vega,~L.~D.}; Moore, M.; Boston, MA. Jul. 2014.

\item \href{http://adsabs.harvard.edu/abs/2014AAS...22345311S}{Diffuse Emission in Nearby, Face-on Spiral Galaxies}; AAS 223rd Meeting; Schlegel, Eric M.; \textbf{Vega, L. D.}; Moore, M.; Washington, DC, Jan. 5-9, 2014.

\item \href{http://adsabs.harvard.edu/abs/2012AAS...21934630V}{Diffuse X-ray Emission and Star Clusters in Nearby, Face-on Spiral Galaxies}; AAS 219th Meeting. Austin, TX, Jan. 7-12 2012.

\item \href{http://adsabs.harvard.edu/abs/2011AAS...21733813M}{X-ray Eclipses in the 2010 Outburst of the Recurrent Nova U Sco?} \& \href{http://adsabs.harvard.edu/abs/2011AAS...21733812M}{Observations of the 2010 January Outburst of the Recurrent Nova U Sco using NASA's Swift}; AAS 117th Meeting; McMaster, Laura; \textbf{Vega, L.}, et al.; Seattle, WA, Jan. 9-13 2011. 


% \item \textbf{Vega, Laura D.}; Branum, A.; Burns, C.; Hoffman, N.; Kim, H.; McMaster, L.; Moore, M.; Schlegel, E. M.; “Diffuse X-ray Emission and Star Clusters in Nearby, Face-on Spiral;” South Central Conference for Undergraduate Women in Physics, SCUWiP. Austin, TX, Jan. 18-20 2013. 

%\item McMaster, Laura; \textbf{Vega, L.}; Moore, M.; Engelhardt, S.; Schlegel E.; Pagnotta, A.; “Observations of the 2010 January Outburst of the Recurrent Nova U Sco using NASA's Swift” \& “X-ray Eclipses in the 2010 Outburst of the Recurrent Nova U Sco?” SCUWiP. College Station, TX, Jan. 13-15 2012.

% \item \textbf{Vega, Laura D.}; Marzal, P.; Garcia-Martinez, I.; Cruz-Gonzalez, N.; Larios-Rodriguez, E.; Yacaman, M. J.;  “Molybdenum Sulfide at a Nanoscale: Structure and Catalytic Analyses;” UTSA College of Science, COS, Research Conference. San Antonio, TX, Oct. 5, 2012.

%\item McMaster, Laura; \textbf{Vega, L.}; Moore, M.; Engelhardt, S.; Schlegel E.; Pagnotta, A.; “X-ray Eclipses in the 2010 Outburst of the Recurrent Nova U Sco?” AND “Observations of the 2010 January Outburst of the Recurrent Nova U Sco using NASA's Swift;” UTSA COS Research Conference. San Antonio, TX, Sep. 20, 2011.

\item \href{http://adsabs.harvard.edu/abs/2010APS..TSFFP1019V}{“X-ray Eclipses in the 2010 Outburst of the Recurrent Nova U Sco?”} \&  \href{http://adsabs.harvard.edu/abs/2010APS..TSFFP1013M}{“Observations of the 2010 January Outburst of the Recurrent Nova U Scoprius using NASA's Swift”}; Joint Fall 2010 Meeting of the Texas Sections of the APS, AAPT, Zone 13 of SPS and the National Society of Hispanic Physicists, San Antonio, TX, October 21-23, 2010.

\end{etaremune}

% \end{enumerate}

% \bigskip

\noindent
{\bf Leadership \& Outreach} \\
\vspace{-9mm}
\begin{center}
\rule{\textwidth}{0.2mm}
\end{center}
\vspace{-3mm}

\noindent
\textbf{Better Astronomy for the New Generation! (BANG!) Organizing Committee Member, Sept. 2022 - present} \\
- The University of Maryland's Astronomy Department's BANG! Seminar features speakers and discussions on issues of equity and inclusion, career paths, and soft skills for astronomers.
\vskip 0.2in

\noindent
\textbf{Research Mentor, Oct. 2020 - Jan 2021} \\
- Mentoring a high school senior in Madison County, VA, developing a research project on classifying M-dwarf flares based on the Solar-flare classification system.
\vskip 0.2in

\noindent
\textbf{Colorado Springs Cool Science Festival - Zoom-a-Scientist, Oct. 2020} \\
- Virtual (Zoom) chat sessions with children in Fairbanks, AK (3rd grade) and in Colorado Springs, CO (1st grade) about astronomy and what it's like to be a scientist.
\vskip 0.2in

\noindent
\textbf{Smithsonian National Air \& Space Museum Volunteer Astronomer, Oct.~2017~-~Present}\\
- \href{https://airandspace.si.edu/events/astronomy-chat-gabriella-alvarez-and-laura-vega-0}{Astronomy Chats}
% Museum-goers can video-chat or chat with me in person and ask questions about my research and what it is like to be an astronomer. \\
\& Hispanic Heritage Month Outreach: Innovators in Air and Space Family Day, Wash. DC.\\
% \vskip 0.1in

\noindent
\textbf{Barrera Veterans Elementary School - Career Day, Von Ormy, TX, May 2019 } \\
- Chat with children at different grade levels about astronomy and what it's like to be a scientist.
\vskip 0.2in

\noindent
\textbf{Bancroft Elementary Saturday Academy - Astronomy Chat, Wash. DC, January 2019} \\
- Visited 2nd and 3rd grade Latinx bilingual students at Bancroft Elementary to talk about astronomy and science. Presentation facilitated by the Smithsonian Air \& Space Museum and sponsored by the Smithsonian Latino Center. \\
\vskip 0.01in

\noindent 
\textbf{Smithsonian Astrophysical Observatory - Rising Stars Camp, St. Katharine Drexel, Roxbury Crossing, MA, July 2018} \\
- Facilitated a number of activities such as the use of the Micro Observatory Robotic Telescopes and software to introduce the children to the life cycle of stars, and other astronomy topics. I also gave a short talk about my research.  \\
\vskip 0.01in

\noindent
\textbf{Fisk-Vanderbilt Bridge Program Recruitment at SACNAS, Long Beach, CA, Oct 2016} \\
- Booth at SACNAS to recruit undergraduate student for the Fisk-Vanderbilt Bridge Program. \\
% - Attended the National Society for Hispanic Physicists. \\

\vskip 0.01in
\noindent
\textbf{DCA Stars at The University of Maryland - College Park, Summer 2016} \\
- Students and mentors met weekly and discussed our research, goals, setbacks, and successes. \\

\vskip 0.01in
\noindent
\textbf{Inclusive Astronomy Conference at Vanderbilt, Nashville, TN, June 17-19, 2015} \\
- Volunteer host for the inaugural conference on diversity and inclusion. \\

\vskip 0.01in
\noindent
\textbf{Astronomy Torus 2012 Workshop at UTSA, San Antonio, TX, December 5-7, 2012} \\
- International workshop focused on active galactic nuclei; Local Organizing Committee Member: Logistics planning; Registration desk. \\

\vskip 0.01in
\noindent
\textbf{Toastmasters International, 10/2013- Present} \\
- Vanderbilt Toastmasters Vice President of Public Relations (05/2015-Present); UTSA Roadrunner Toastmasters Treasurer (01/2014-07/2014); Deliver prepared and extemporaneous speeches as well as evaluating other speakers. \\

\vskip 0.01in
\noindent
\textbf{Society for Advancement of Chicanos \& Native Americans in Science (SACNAS)} \\
- Founding officer (Treasurer) for the {\it first} SACNAS chapter in the state of Tennessee. \\
% - Hosted an abstract workshop for the 2015 SACNAS Conference. Gave tips and advice on how to make a good scientific abstract. \\

\vskip 0.01in
\noindent
\textbf{The Physics Honor Society \text{\textbar} Sigma Pi Sigma at UTSA – President, 04/2013--01/2014} \\
- Organized student study sessions for the Physics GRE; Tutored students in physics and Mathematics; Supplemental officer for Society of Physics Students: Public Relations \& Secretary. \\

\vskip 0.01in
\noindent
% \newpage
\textbf{Society of Physics Students at UTSA} \\
- \textbf{\textit{Outreach}}: Various on-campus STEM outreach events for high school students like Nano Day with Partnership for Education \& Research in Materials at UTSA \& Hispanic Chamber of Commerce’s CORE4 STEM Familia Day Expo at UTSA; Space Day for elementary school children at Forest Hills Public Library; Student Panelist for Engineering-Day at UTSA; On-campus liquid-nitrogen ice cream sales with physics demonstrations. \\
- \textbf{\textit{Public Relations Officer}}: Made posters and flyers of meetings, fundraisers, talks, and events; Managed social media accounts such as SPS’ Facebook page; UTSA Physics \& Astronomy`s “Friday Nights, Celestial Lights” monthly outreach event. \\
% \indent - Promoted the visit of Nobel Laureate John C. Mather and of Sir Roger Penrose to UTSA \\
% \end{list}

% \bigskip

% \newpage
 
\noindent
{\bf Professional Memberships/Affiliations} \\
\vspace{-10mm}
\begin{center}
\rule{\textwidth}{0.2mm}
\end{center}
\vspace{-3mm}
\noindent
Vanderbilt Association of Hispanic and Latinx Alumni \hfill 2021 - Present \\
Sigma Xi \text{\textbar} The Scientific Research Society ($\Sigma \Xi$) \hfill 2015 - Present \\
National Society of Hispanic Physicists (NSHP) \hfill 2013 - Present \\
Toastmasters International \hfill 2013 - Present \\
Society for Advancement of Chicanos \& Native Americans in Science (SACNAS) \hfill 2013 - Present  \\
American Physical Society (APS) \hfill 2011 - Present \\
American Astronomical Society (AAS) \hfill 2011 - Present \\
Sigma Pi Sigma \text{\textbar} The Physics Honor Society ($\Sigma \Pi \Sigma$) \hfill 2011 - Present \\
LeaderShape Institute Alumna \& Pebble Club \hfill 2011 - Present \\
UTSA Alumni Association \hfill 2013 - 2014 \\
San Antonio Astronomical Association \hfill 2013 - 2014 \\
UTSA San Antonio Texas Astrophysics Researchers (STAR) \hfill 2011 - 2014 \\
Society of Physics Students (SPS) \hfill 2010 - 2013 \\
National Society of Collegiate Scholars (NSCS) \hfill 2008 - 2013 \\

% \bigskip

% \newpage
% \noindent
% {\bf References} \\
% \vspace{-10mm}
% \begin{center}
% \rule{\textwidth}{0.2mm}
% \end{center}
% \vspace{-3mm}

% \noindent
% \href{http://astro.phy.vanderbilt.edu/~stassuk/}{\textbf{Keivan G. Stassun}}, Stevenson Professor of Physics and Astronomy \\
% Vanderbilt University, Nashville, TN \\
% (615) 322-2828 \\
% keivan.stassun@vanderbilt.edu \\
% \textit{Dissertation Advisor} \\

% \noindent
% \href{http://www.rudyphd.com/}{\textbf{Rodolfo Montez, Jr.}}, Astrophysicist \\
% Smithsonian Astronomical Observatory, Cambridge, MA \\
% (617) 496-7565 \\
% rodolfo.montez@cfa.harvard.edu \\
% \textit{Research Mentor, Dissertation Committee Member} \\

% \noindent
% \href{https://www.utsa.edu/physics/faculty/EricSchlegel.html}{\textbf{Eric M. Schlegel}}, Vaughan Family Professor of Astronomy \\
% The University of Texas at San Antonio, San Antonio, TX \\
% (210) 458-6425 \\
% eric.schlegel@utsa.edu \\
% \textit{Dissertation Committee Member, Former Undergraduate Research Advisor} \\

% \noindent
% \href{https://science.gsfc.nasa.gov/sed/bio/patricia.t.boyd}{\textbf{Patricia T. Boyd}}, Director, NASA SMD Bridge Program \\
% Former Chief, Exoplanets and Stellar Astrophysics Laboratory \\
% NASA Goddard Space Flight Center, Greenbelt, MD \\
% (301) 286-2550 \\
% patricia.t.boyd@nasa.gov \\
% \textit{ NASA Mentor, Dissertation Committee Member} \\

\end{document}
